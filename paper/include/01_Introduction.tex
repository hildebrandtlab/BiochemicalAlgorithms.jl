\section{Introduction}


The aim of structure-based bioinformatics is the analysis and targeted manipulation of three-dimensional structures of biological macromolecules such as proteins and nucleic acids. This research field integrates disciplines ranging from fundamental physical laws to complex biochemistry knowledge and advanced numerical computing methodologies. For example, in molecular mechanics, a molecular force field is used to compute the energy of a structure.  \\
Structure-based bioinformatics encompasses applications such as molecular modeling, molecular dynamics (MD) simulations, and molecular docking.  Molecular modeling techniques, particularly docking suites, attracted widespread interest during the COVID-19 pandemic: In the early phase, the protein structures were predicted based on sequence data as experimentally verified structures were not yet available. These structures were examined, including the analysis of the effects of the mutations originating from various virus strains. The knowledge of these molecular functions was then used in the context of rational drug design to find potential vaccines or drug therapeutics \cite{Kumar2021}. \\
The COVID-19 pandemic highlighted the importance of these applications. However, most open-source software packages were developed much earlier, between 1995 and 2010. For instance, the molecular docking tools \textit{AutoDock Vina} and its predecessor \textit{AutoDock4} regained considerable popularity during the pandemic; however, they were developed much earlier in 2009 and 2010 \cite{Trott2010, Morris2009AutoDock4AA}.\\
In the last decade before the pandemic, there have been no significant innovations in structure-based software developments. Several reasons contribute to the decreasing interest in this field. A crucial aspect is the availability of molecular structure data. Historically, the number of experimentally resolved structures was limited for many years \cite{berman2000protein} leading to a slowdown of the progress in this area. This changed in 2018 when DeepMind entered the CASP competition with AlphaFold \cite{deepmind} and, hence, put structure prediction in the spotlight again. Additionally, the number of experimentally determined structures with high resolution has increased dramatically through advances in cryo-electron microscopy in recent years. Nowadays, with the rapid rise of computed structures, availability no longer restrains the development of structural bioinformatics applications \cite{AlphaFoldDB2023}. \\

Software development in this field has been -- and still is -- typically challenging due to its interdisciplinary nature. The need for both numerical stability and computational efficiency, along with ease of use, has been a significant obstacle to rapid application development (RAD) in open-source projects. 
Schroedinger is a closed-source framework providing packages for different molecular applications: \texttt{ProteinPreparationWizard} deals with the preprocessing of protein structural models and \texttt{LiveDesign} focuses on docking and designing ligands \cite{SchrödingerPPW, SchrödingerLiveDesign}. These tools have the undeniable disadvantage of being closed-source and not free of charge. \\
For many open-source software packages, it is not uncommon to focus on implementing one specific task or algorithm (e.g., the introduction of a docking algorithm). The drawback of this approach is, that the user has to virtually glue several tools together providing the specific functionality. For instance, the structures have to be properly preprocessed before they can be used as input for a docking algorithm but these tools usually lack functionalities for preprocessing.  \\
An exception to this single-purpose approach is the introduction of \ballFull (\ball) by Kohlbacher \textit{et al.} in 1996. \ball\ is a well-designed framework for molecular structure analysis written in C++. It offers file import and export, structure preprocessing, molecular mechanics, advanced solvation methods, and visualization options. Because of many contributions at the time \ball\ used to have one of the biggest user communities in this field. In 2010, a new version introduced Python bindings for enhanced usability.\\

A package for molecular dynamics simulation was published in more recent times and, similar to \ball\, it was written in C++ and included additional Python bindings \cite{Doerr2016HTMD}. While C++ is a natural choice to achieve the required efficiency of programs, it effectively hinders the rapid prototyping of molecular algorithms.  \\
Developing software for structural bioinformatics is still challenging; however, choosing the programming language is not. Julia offers both the efficiency and numerical stability needed for molecular simulations along with rapid development capabilities. Several packages already exist in Julia related to structural bioinformatics -- most prominently under two Github communities \textit{Molecular Simulation in Julia} and \textit{BioJulia} \cite{JuliaMolSim, BioJulia}. The latter offers software packages for general biology approaches such as \textit{BioSymbols.jl} for the representation of nucleic and amino acid primitive types as well as packages related to sequential bioinformatics. Most notably, it provides \textit{BioStructures.jl} for reading and writing of macromolecular structures \cite{Greener2020BioStructures}. Additionally, Greener \textit{et al.} published \textit{Molly.jl}, a package for molecular simulations, which is part of the \textit{Molecular Simulation in Julia} Github community \cite{Greener2024}.
Another interesting approach is \textit{ProtoSyn.jl}; although it provides functionalities for analyzing peptides, its main focus is restricted to peptide design and engineering. \\
The mentioned Julia packages are limited to specific tasks e.g., \textit{BioStructures.jl} is an excellent package for reading and writing PDB files. However, there remains a need for a platform that acts as an entry point by offering functionality encompassing an entire molecular structure pipeline.\\
We present \biochem\ as a general-purpose framework for structure-based bioinformatics. \biochem\ is a redesign of \ball\ and provides the foundation for molecular modeling and molecular simulation studies. Currently, we provide functionalities for:
\begin{itemize}
	\item reading common data formats such as PDB, PDBx/mmCIF, HyperChem HIN, SDF (Structured Data File), and PubChem JSON
	\item preprocessing the input by preparing the entire system (e.g., adding missing hydrogens, bond computation, reconstruction of missing atoms, \dots)
	\item molecular mechanics through AMBER force fields 
	\item mapping of structures
	\item structure minimization
	\item visualization of structures with \bioviz
\end{itemize}
In addition, \biochem's intuitive interfaces enable users to develop their custom applications like the implementation of force fields for specific needs. \\

This article is organized as follows: First, we give a short background on \ball\ because this C++ framework motivated our design. In the next section, we depict how the switch from C++ to Julia improved our development. Thereafter, \biochem's core is described. The ease of use and functionality are showcased in the application section. It contains four use cases including a comparison of C++ and Julia as well as visualizations. 