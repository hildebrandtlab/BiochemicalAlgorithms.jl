
\section{Conclusion}

We have presented a single and modern platform for the rapid application development of molecular structure analysis and simulations. While other Julia packages focused either on a single task like the manipulation of single peptides or molecular dynamic simulations, \biochem serves as framework for the explorative studies of structure analysis or the proper initialization of structures for molecular simulations through the usage of the fragmentDB. Like its C++ predecessor, \biochem provides a robust but flexible core with additional functionalities. More precisely, the core has been carefully designed using a costume realization of the \texttt{Tables.jl}-interface. Thereby, our platform 
\begin{itemize}
	\item can be integrated to other tools using the tables interface
	\item fits into julia mol sim or BioJulia
	\item Achievement: \biochem allows the rapid prototyping of molecular analysis 
	\item Meaning of achievement: interface is so flexible that parts can be interchanged or it is easy to write your own tool not to take care of proper initialization if yout want to implement a molecular force field 
	\item NOVELTY: Do we have one? 
	\item We believe that \biochem will be very helpful for scientist who want to do structural bioinformatics and don't know much about....additionally: visualization
	\item \biochem already provides basic functionalities and interfaces for molecular mechanics but much is still to be done e.g., implementation of a docking interface, implementation of other force fields... we still have to do a lot of stuff to be feature-complete to \ball
	\item 
\end{itemize}